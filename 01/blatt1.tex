\documentclass[a4paper,url]{article}


\usepackage[utf8]{inputenc}
\usepackage[T1]{fontenc}
\usepackage[german]{babel}
\usepackage[small]{subfigure}
\usepackage{amsmath}
\usepackage{enumerate}
\usepackage{halloweenmath}
\usepackage{graphicx}
\usepackage{ifthen}
%\usepackage{times, courier, helvet, mathpple} %% for nice PDF fonts
\usepackage{amssymb}
\usepackage{hyperref}


% Change dimension of a page
\usepackage{geometry}
\geometry{a4paper, left = 30mm, right=30mm, top=30mm, bottom=30mm, headheight=1mm, headsep=1mm, footskip=15mm}



% Draw with tikzpicture
\usepackage{pgfplots}
% line thickness: ultra thin, very thin, thin, semithick, thick, very thick, ultra thick
% line style:     dashed, loosely dashed, densely dashed, dotted, loosely dotted, densely dotted
\tikzset{vertice/.style={circle, draw, minimum size=7mm, scale=0.8, fill=black!15}}
\tikzset{edge/.style={-, black}}






\newcommand {\rpf}{\begin{math}\rightarrow\end{math}}
\newcommand {\ra}{\rightarrow}
\newcommand {\epsi}{\begin{math}\epsilon\end{math}}

\selectlanguage{german}

\begin{document}
\newcommand{\nat}{\mbox{I}\!\mbox{N}}

\newcommand{\real}{\mbox{I}\!\mbox{R}}

\newcounter{aufgabe_count}
\setcounter{aufgabe_count}{1}
\newcommand{\aufgabe}[2]{\vspace{3.5ex} {\noindent \bf\large Aufgabe
\arabic{aufgabe_count}: \hspace{10pt}#1} \hspace{5pt}(#2 Punkte)\vspace{3pt}\\ 
\stepcounter{aufgabe_count} 
}

%\topmargin0in

\textheight55\baselineskip

\pagestyle{plain}
\pagenumbering{arabic}

\noindent
\begin{minipage}[t]{0.6\textwidth}
\begin{flushleft}
\bf Übungen zur Vorlesung\\
Grundlagen des Internets\\
Prof. Dr. Michael Menth
\end{flushleft}
\end{minipage}
\begin{minipage}[t]{0.4\textwidth}
\begin{flushright}
\bf SS 2019\\
Universität Tübingen\\
\today %@@@Datum eintragen
\end{flushright}
\end{minipage}



\vspace{5.0ex}
\noindent

\centerline{\huge \bf Übungsblatt 1}
\centerline{\bf Lukas Günthner, 4124568, lukas.guenthner@student.uni-tuebingen.de}
\centerline{\bf Vinhdo Doan, vinhdo.doan@student.uni-tuebingen.de}
	\centerline{(War leider am Morgen der Abgabe nicht zu erreichen, deshalb fehlt die MatrklNr noch)}
%%%%%%%%%%%%%%%%%%%%%%%%%%%%%%%%%%%%%%%%%%%%%%%%%%%%%%%%%%%%%%%%%%%%%%%%%%%%%%%%%%%%%%%%%%%%%%%%%%%
\section*{1.1.1}
\begin{itemize}
	\item IP: 3
	\item TCP: 4
	\item Twisted-Pair Kabel: 1
	\item DNS: 7
	\item ARP: 2
	\item Token ring: 2
	\item RTSP: 7
\end{itemize}

\section*{1.1.2}
\begin{itemize}
	\item Router: 3
	\item IEE 802.11: 1
	\item IEE 802.3: 1+2
	\item Bridge: 2
	\item Browser: 7
	\item Hub: 1
	\item PPP: 2
	\item MPLS: 2+3
\end{itemize}

\section*{1.1.3}
Physical + Data Link werden zu Network Access Schicht. Das Session und Presentation Layer sind nicht mehr im TCP/IP Schichtenmodell vorhanden.
Applikationen dürfen direkt auf IP aufsetzen also dem Internet Layer im TCP/IP Modell.

\section*{1.2.1}
Das Internet ist ein zusammenschluss aus verschiedenen Netzwerken (LAN's). Zum Beispiel wird das Heimnetzwerk über das ISP Netzwerk mit den restlichen Netzen im Internet zusammengschlossen.

\section*{1.2.2}
Local ISP stellt den Zugang zum Internet für ein lokales Netzwerk (Heimnetz) bereit.
Regional ISP's stellt dann den Zugang zum überregionalen Internet für die lokalen ISP's bereit.

\section*{1.2.3}
\begin{itemize}
	\item local ISP: Router
	\item regional ISP: 1\&1
\end{itemize}

\section*{1.2.4}
\begin{align}
	\frac{n\cdot h}{M + (n \cdot h)}
\end{align}
Es wird die Datenmenge die für alle Header verwendet wird durch die komplette Datenmenge (Daten + Header) geteilt.

\section*{1.3.1}
\begin{itemize}
	\item A: Layer 2, MAC
	\item B: Layer 2, MAC
	\item C: Layer 3, IP
	\item D: Layer 4, TCP
\end{itemize}

\section*{1.3.2}
\begin{tabular}{c|c|c|c|c}
	& (A) & (B) & (C) & (D) \\
	\hline
	\hline
	Protokoll Schicht 4 & & & & TCP \\
	\hline
	Protokoll Schicht 3 & & & IP & \\
	\hline
	Protokoll Schicht 2 & MAC & MAC & & \\
	\hline
	\hline
	Zielport (Schicht 4) &  &  & & \\
	\hline
	Quelladresse Schicht 3 & 193.196.20.254 &  & & 134.2.3.254\\
	\hline
	Zieladresse Schicht 3 & 193.196.20.254 &  & & 134.2.3.1 \\
	\hline
	Quelladresse Schicht 2 & 3f:d2:5e:86:aa:0a & d5:08:f4:67:38:95 & ce:34:30:97:b4:1e & 75:c4:3c:eb:2d:fa\\
	\hline
	Zieladresse Schicht 2 & d5:08:f4:67:38:95 & ce:34:30:97:b4:1e & fb:30:89:60:35:0e & 9d:76:bc:e7:00:b1\\
\end{tabular}















\end{document}

























