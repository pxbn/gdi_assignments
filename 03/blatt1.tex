\documentclass[a4paper,url]{article}


\usepackage[utf8]{inputenc}
\usepackage[T1]{fontenc}
\usepackage[german]{babel}
\usepackage[small]{subfigure}
\usepackage{amsmath}
\usepackage{enumerate}
\usepackage{halloweenmath}
\usepackage{graphicx}
\usepackage{ifthen}
%\usepackage{times, courier, helvet, mathpple} %% for nice PDF fonts
\usepackage{amssymb}
\usepackage{hyperref}


% Change dimension of a page
\usepackage{geometry}
\geometry{a4paper, left = 30mm, right=30mm, top=30mm, bottom=30mm, headheight=1mm, headsep=1mm, footskip=15mm}



% Draw with tikzpicture
\usepackage{pgfplots}
% line thickness: ultra thin, very thin, thin, semithick, thick, very thick, ultra thick
% line style:     dashed, loosely dashed, densely dashed, dotted, loosely dotted, densely dotted
\tikzset{vertice/.style={circle, draw, minimum size=7mm, scale=0.8, fill=black!15}}
\tikzset{edge/.style={-, black}}






\newcommand {\rpf}{\begin{math}\rightarrow\end{math}}
\newcommand {\ra}{\rightarrow}
\newcommand {\epsi}{\begin{math}\epsilon\end{math}}

\selectlanguage{german}

\begin{document}
\newcommand{\nat}{\mbox{I}\!\mbox{N}}

\newcommand{\real}{\mbox{I}\!\mbox{R}}

\newcounter{aufgabe_count}
\setcounter{aufgabe_count}{1}
\newcommand{\aufgabe}[2]{\vspace{3.5ex} {\noindent \bf\large Aufgabe
\arabic{aufgabe_count}: \hspace{10pt}#1} \hspace{5pt}(#2 Punkte)\vspace{3pt}\\ 
\stepcounter{aufgabe_count} 
}

%\topmargin0in

\textheight55\baselineskip

\pagestyle{plain}
\pagenumbering{arabic}

\noindent
\begin{minipage}[t]{0.6\textwidth}
\begin{flushleft}
\bf Übungen zur Vorlesung\\
Grundlagen des Internets\\
Prof. Dr. Michael Menth
\end{flushleft}
\end{minipage}
\begin{minipage}[t]{0.4\textwidth}
\begin{flushright}
\bf SS 2019\\
Universität Tübingen\\
\today %@@@Datum eintragen
\end{flushright}
\end{minipage}



\vspace{5.0ex}
\noindent

\centerline{\huge \bf Übungsblatt 1}
\centerline{\bf Lukas Günthner, 4124568, lukas.guenthner@student.uni-tuebingen.de}
\centerline{\bf Vinhdo Doan, vinhdo.doan@student.uni-tuebingen.de}
	\centerline{(War leider am Morgen der Abgabe nicht zu erreichen, deshalb fehlt die MatrklNr noch)}
%%%%%%%%%%%%%%%%%%%%%%%%%%%%%%%%%%%%%%%%%%%%%%%%%%%%%%%%%%%%%%%%%%%%%%%%%%%%%%%%%%%%%%%%%%%%%%%%%%%
\section*{Aufgabe 3.3}
\subsection*{3.3.1}
Der ARP-Request besteht aus einem Layer-1 Header und der tatsächlichen Layer-2 ARP-Request. Der ARP-Header enthält folgende Felder:
\begin{itemize}
	\item Hardware type, gibt an welches Protokoll auf dem Link-Layer verwendet wird.
	\item Protocol type, gibt an für welches Internetprotkoll die ARP-Request gesendet wird.
	\item Hardware size, Größe der MAC-Adresse
	\item Protocol zsie, Größe der IP-Adresse
	\item Opcode, Operation code 1 für request, 2 für response
	\item Sender MAC adress, eigene MAC-Adresse
	\item Sender IP adress, eigene IP-Adresse
	\item Target MAC adress, bei request 00:00:00:00:00:00 da broadcast
	\item Target IP adress, Ziel IP-Adresse
\end{itemize}

\subsection*{3.3.2}
Der ARP-Response besteht aus einem Layer-1 Header und der tatsächlichen Layer-2 ARP-Response. Der ARP-Header enthält folgende Felder:
\begin{itemize}
	\item Hardware type, gibt an welches Protokoll auf dem Link-Layer verwendet wird.
	\item Protocol type, gibt an für welches Internetprotkoll die ARP-Request gesendet wird.
	\item Hardware size, Größe der MAC-Adresse
	\item Protocol zsie, Größe der IP-Adresse
	\item Opcode, Operation code 1 für request, 2 für response
	\item Sender MAC adress
	\item Sender IP adress
	\item Target MAC adress
	\item Target IP adress
\end{itemize}

\subsection*{3.3.3}
%TODO: TODODOODODODODODOODODODODODODOODODODODODOODODODODODODODO

\subsection*{3.3.4}
Ja, da eine ARP-Request ans ganze Netzwerk gebroadcastet wird und keine Sicherheitsmechanismen (z.B. Authentifizierung) implementiert, ist es anfällig für bspw. Spoofing und kann damit für man-in-the-middle attacken ausgenutzt werden.

\subsection*{3.3.5}
Nein, da alle Hosts alle ARP-Replies in ihre ARP-Tabelle abspeicher, egal ob sie überhaupt eine ARP-Request verschickt haben oder nicht. \footnote{Siehe \href{https://en.wikipedia.org/wiki/ARP_spoofing#ARP_vulnerabilities}{https://en.wikipedia.org/wiki/ARP\_spoofing\#ARP\_vulnerabilities}}

\subsection*{3.3.6}
Es müssen zwei ARP-Pakete versendet werden.
\begin{enumerate}
	\item A broadcast 'Who is B' (request)
	\item B antwortet mit seiner MAC-Adresse (reply)
\end{enumerate}
Zuerst schickt A den ARP-Rrquest broadcast, B schickt dann seine ARP-Reply mit seiner MAC-Adresse und A speichert diese in seiner ARP-Tabelle. C empfängt diese Reply auch und speichert genauso die MAC-Adresse von B in seiner ARP-Tabelle.

\end{document}

























